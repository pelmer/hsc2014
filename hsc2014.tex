\documentclass[12pt,letterpaper,fleqn]{article}
\usepackage{graphicx}
\usepackage{lineno}
\usepackage{hyperref}
\usepackage{subfigure}
\usepackage{cite}
\usepackage{xcolor,colortbl}
\usepackage{wrapfig}
\usepackage{enumitem} %%% uncommenting this line and associated
%% options only saves about 1/4 page
% enumitem options
%\setlist{nosep}
%\setitemize{wide}
%\setlist[itemize]{leftmargin=*}
%\setlist[enumerate]{leftmargin=*}


\definecolor{Gray}{gray}{0.85}

\textwidth      6.65 in
\textheight     9.0 in
\topmargin      0.5 in
\oddsidemargin  0.0 in
\evensidemargin 0.0 in
\headheight =  -1.0 in  

\author{Lothar Bauerdick, Paolo Calafiura, Peter Elmer, Liz Sexton, \\
Panagiotis Spentzouris, Torre Wenaus, Avi Yagil}
\title{HEP Software Consortium}

\begin{document}
\maketitle
\linenumbers

\section{Objective}
\label{sec:objective}


HEP computing is facing many challenges ahead, including the High
Luminosity LHC program in Europe, the Intensity Frontier (IF) program
in the US and the evolution of computing hardware technology.
Significant resources are required to maintain and further develop
the scientific computing infrastructure needed by the current and
future HEP programs.  Data analysis campaigns last for many years
(decades!)\ and involve hundreds of developers and experimenters.
The data volumes and processing needs for the LHC continue to
increase  while the many IF
experiments will be in various phases of design and operations in
the next decade.  In addition, the evolution of the technical
landscape calls for major software re-engineering. Both the
physics and the technology {\em require more sophisticated software
tools}.

At the same time, overall HEP funding has been declining in the past
few years. {\em Multiple independent efforts to build this software
will prove both more costly and ultimately produce less
sophisticated and less sustainable software.} It is thus
desirable to take advantage of commonality in
needs of experiments and leverage expertise across all programs and
projects. While some organizational structures exist within individual
labs, between experiments at a given lab or in specific software domains,
no general framework exists for collaborative software efforts across the entire HEP field.

In this document we describe a proposal for an HEP Software Consortium
(HSC) whose aim is to {\em foster the development of a high quality,
innovative, efficient and sustainable software ecosystem of general
utility for the HEP community.}





\section{Software Consortium}
\label{sec:consortium}

This HEP Software Consortium
(HSC) would aim to identify areas that benefit from collaboration
between individual projects and to facilitate the sharing/pooling
of expertise and resources to solve common problems.  It will
organize forums to discuss best practice and software development
issues and propose better software engineering solutions. 
%Eventually, the HSC could evolve to also incorporate an HSC centric project.
Because many of our tools have broader applicability for science
outside HEP, the existence of such a project could help generate
new funding opportunities outside our traditional funding ``sandbox''
(funding agencies with narrow HEP focus) by increasing the visibility
and appeal of our scientific software activities.

There is a plethora of mature and successful HEP software projects,
each with different models of organization, established processes,
programs, and sponsors.  The HSC will provide an effective mechanism
for these projects to connect and collaborate, while maintaining
their separate and independent entities.  We envision the HSC as a
partnership of projects, which explores the common needs and interests
of the participants, identifies community goals and deliverables,
and facilitates community contributions. Consortium participants
will come initially from HEP scientific software projects (e.g.
Root, Geant4, GENIE, xrootd, etc), HEP experiments, and university and
laboratory scientific software and computer science groups.  In the
future, if the HSC enables and fosters outreach to other communities
beyond HEP, participation could be expanded to similar groups from
non-HEP domains.

The Consortium will involve all partners in the design, development
and possibly deployment process to ensure that the HSC will function
and evolve consistently while meeting the needs of the individual
partners.  The Consortium will sponsor activities (workshops,
meetings, discussion forums) to facilitate the development of
software engineering standards and software architecture guidelines
by identifying common needs and encouraging common solutions. The
participating institutions and partner projects will contribute to
the Consortium by providing input and expertise in defining common
goals and developing such principles and guidelines.  In addition,
they will contribute to the HSC software infrastructure by developing
and deploying software within the common development and testing
guidelines and following the HSC software architecture principles.
This will include adapting existing software tools to the HSC
standards, when possible.  The partner software projects will
maintain their
independent organization frameworks and have ownership of the
software they are developing and deploying utilizing the common
standards.  The HSC will publish the code that complies with the
standards and organize peer-reviews for the software that is developed
under its process and guidelines.  Specific examples of HSC activities
could include:
\begin{itemize}
\item sharing technical expertise in architecture and design issues,
\item the development of standards for component interfaces and layout, to
enable independent development and easy component integration,
\item facilitating the development of non-domain specific components
(geometry representation, data representation) that can be used
by many projects,
\item developing standards and guidelines for testing releasing and
documenting software,
\item proposing solutions for distributed collaborative environment
and common infrastructure such as build tools, testing and validation
suites,
\item maintaining a list of HEP software products, and
\item providing information as to whether the software packages follow
architectural and engineering recommendations developed in the HSC.
\end{itemize}

Note that the initiative (and buy-in) for these activities
must come from motivated
individuals, software projects, institutions and/or experiments within
the HSC. It is not intended to be a ``top-down'' process driven
entirely by the chair of one or another governance entity, but
a structured community process. In order for this program of work
to be successful, the Consortium will also adopt an ``open-source''
software model and rely on agile development methods.  

%The organization framework of the Consortium
%will include Technical Domain Forums each with responsibility in a
%technical domain such as simulation, analytics, frameworks, and
%crosscutting R\&D activities (the concurrency forum is a good example).
%The Forums will facilitate the organization of workshops and meetings,
%document and distribute the findings (wikis, blogs, etc).   


\section{Governance}
\label{sec:governance}

From the discussions at the workshop at CERN on 3-4 April, 2014, it
is clear that the community in general prefers a lightweight
organization, whose successes will be driven in a ``bottoms-up''
fashion. Indeed fostering a software ecosystem is a different
type of activity from a detector construction project or a traditional
HEP experimental collaboration. We describe here the functional
entities of such a lightweight ``governance'' that would provide
the basic mechanisms for the consortium to achieve the overall
aim of fostering the development of a high quality,
innovative, efficient and sustainable software ecosystem of general
utility for the HEP community.

{\bf Software Engineering Board (SEB):} The activities of the SEB provide the primary ``software ecosystem'' functionality of the Consortium.  
The SEB provides expertise and advise on software architecture and engineering issues, organizes and contributes to peer-reviews,
discusses and proposes best practices and develops guidelines.
The members of the board play a leading role in the activities of
the Technical Domain Forums (described below) by organizing and chairing the Forum discussions and meetings, and discussing the outcome
of the Forum activities at the Board meetings. In essence, it should 
play a role similar to the LHC ``Architects Forum'', with clearer
mechanisms for interacting with major stakeholders (Consortium Council
below).
The membership of this board is {\em inclusive and open}. In general each software project and/or major user group (e.g.\ experiments) simply self-identifies a representative to participate. The members of the Board elect a fixed term chair to moderate discussions and foster consensus.
The ultimate success of the HSC should be judged on the synergies,
initiatives and activities enabling a stronger software ecosystem
which would not have existed in the absence of the HSC.

{\bf Technical Domain Forums:} provide coordination and facilitate
communication between the participating projects and the consortium
on specific topics. One such example today is the ``Concurrency Forum''.
These could be standing forums on topics of general interest or
temporary working groups on specific topics or initiatives.
To achieve their goals they may organize workshops; maintain wikis,
blogs, etc. The SEB may create new technical forums as needed and pre-existing entities effectively playing this role can simply self-identify
to the chair of the SEB. The chair of the SEB will maintain a list
of such forums, their scope and objectives.

{\bf Consortium Council:} The role of the council is to enable the
work of the Software Engineering Board.
It consists of representatives of
institutions providing resources to the software projects
and major experiments and/or user groups; the stakeholders board.
(We believe that the healthiest starting point
for the HSC is that it {\em not} have resources of its own, independent
from these stakeholders.)
The Council holds infrequent meetings (at least once a year) to 
identify common areas of interest and discuss common goals and overall
direction based on the input from the Advisory Board and the results of the activities of the Software Engineering Board. The Council will
also elect a fixed term chair. 

{\bf Advisory Board:} The role of the Advisory Board is to provide
feedback and recommendations on scientific and technical priorities
and needs to the Consortium Council. It consists of individuals from
the HEP community at large, other scientific communities, and
software industry experts. They are {\em not} chosen as representatives
of particular user groups, experiments, institutions or software
projects, but rather for their general expertise.
Proposals for membership in the Advisory Board are solicited from the Consortium Council. The Advisory Board may be asked to perform general
reviews, or simply provide feedback directly to the Council, depending
on the specific needs in any given time period. 


%\section{Incentives}
%
%A recent study of software packages which have become widely used
%in HEP noted the following common characteristics:
%\begin{enumerate}
%\item A clearly defined individual(s) exists as ``Champion(s)''
%\item {\em Collaborations} between individuals, institutions or experiments are formed early to facilitate development
%\item The software is created in the context of an experiment or driven by people who are also users (``Eat your own dogfood'')
%\item Adherence to or development of useful or recognized (de-facto or documented) standards enables wider use
%%\item Spreading of packages from one project/experiment to another by migration of individuals
%\item ROOT at times served as a de-facto distribution mechanism
%\item By-product of a dedicated R\&D project into new functionalities
%\item Unique functionalities
%\item Adoption by a project after failure of internal efforts
%\end{enumerate}


\section{Summary} 

A HEP Software Consortium will provide a community-wide framework
for leveraging and coordinating ongoing HEP software activities and facilitating the creation of an {\it ecosystem} of software activities.   The proposed HSC will:
\begin{itemize}
\item be based on a primarily ``bottoms-up'' structure
\item provide the means to recognize and encourage the common aspects
of successful projects, facilitate collaboration between partner
projects, and facilitate adoption of successful software engineering
practices,
\item promote sharing of software architecture assistance and expertise,
\item help coordinate exploration of new technologies and provide expertise in their implementation,
\item and eventually enable contact with other scientific fields,
providing access to either resources or software tools and expertise.
\end{itemize}



\newpage

%\bibliographystyle{unsrt}
%\bibliography{hsc2014}

\end{document}
