\documentclass[12pt,letterpaper,fleqn]{article}
\usepackage{graphicx}
\usepackage{lineno}
\usepackage{hyperref}
\usepackage{subfigure}
\usepackage{cite}
\usepackage{xcolor,colortbl}
\usepackage{wrapfig}
\usepackage{enumitem} %%% uncommenting this line and associated
%% options only saves about 1/4 page
% enumitem options
%\setlist{nosep}
%\setitemize{wide}
%\setlist[itemize]{leftmargin=*}
%\setlist[enumerate]{leftmargin=*}


\definecolor{Gray}{gray}{0.85}

\textwidth      6.65 in
\textheight     9.0 in
\topmargin      0.5 in
\oddsidemargin  0.0 in
\evensidemargin 0.0 in
\headheight =  -1.0 in  


\begin{document}
\linenumbers

\section{Foo}
\label{sec:intro}

HEP computing is facing many challenges ahead, including the High
Luminosity LHC program in Europe, the Intensity Frontier (IF) program
in the US and the evolution of computing hardware technology. In
this document we describe a proposal for an HEP Software Consortium
(HSC) whose aim is to foster the development of a high quality,
innovative, efficient and sustainable software ecosystem of general
utility for the HEP community.

Significant resources are required to maintain and further develop
the scientific computing infrastructure needed by the current and
future HEP programs.  Data analysis campaigns last for many years
(decades!) and involve hundreds of developers and experimenters.
The data volumes and processing needs for the LHC continue to
increase (maybe a qualitative number here?) while the many IF
experiments will be in various phases of design and operations in
the next decade.  In addition, the evolution of the technical
landscape calls for major software re-engineering.  Because the
overall HEP funding has been declining in the past few years, it
will be difficult to sustain the necessary effort without increasing
the efficiency of our software development model.   Thus, it is
desirable to evolve this model to take advantage of commonality in
needs of experiments and leverage expertise across all programs and
projects.

In order to facilitate this evolution of our software development
model, we propose to form a consortium of HEP software projects
whose purpose will be to establish a framework for collaboration
on specific software aspects, an ecosystem of software activities
in which new ideas could be formed and communicated, and technology
transitions could be facilitated.  This HEP Software Consortium
(HSC) would aim for identifying areas that benefit from collaboration
between individual projects and for facilitating the sharing/pooling
of expertise and resources to solve common problems.  It will
organize forums to discuss best practice and software development
issues and propose better software engineering solutions.  Eventually,
the HSC could evolve to also incorporate an HSC centric project.
Because many of our tools have broader applicability for science
outside HEP, the existence of such a project could help generate
new funding opportunities outside our traditional funding “sandbox”
(funding agencies with narrow HEP focus) by increasing the visibility
and appeal of our scientific software activities.

There is a plethora of mature and successful HEP software projects,
each with different models of organization, established process,
program, and sponsors.  The HSC will provide an effective mechanism
for these projects to connect and collaborate, while maintaining
their separate and independent entities.  We envision the HSC as a
partnership of projects, which explores the common needs and interests
of the participants, identifies community goals and deliverables,
and facilitates community contributions. Consortium participants
will come initially from HEP scientific software projects (e.g.
Root, Geant4, GENIE, etc), HEP experiments, and university and
laboratory scientific software and computer science groups.  In the
future, if the HSC enables and fosters outreach to other communities
beyond HEP, participation could be expanded to similar groups from
non-HEP domains.

The Consortium will involve all partners in the design, development
and possibly deployment process to ensure that the HSC will function
and evolve consistently while meeting the needs of the individual
partners.  The Consortium will sponsor activities (workshops,
meetings, discussion forums) to facilitate the development of
software engineering standards and software architecture guidelines
by identifying common needs and encouraging common solutions. The
participating institutions and partner projects will contribute to
the Consortium by providing input and expertise in defining common
goals and developing such principles and guidelines.  In addition,
they will contribute to the HSC software infrastructure by developing
and deploying software within the common development and testing
guidelines and following the HSC software architecture principles.
This will include adapting existing software tools to the HSC
standards, when possible.  The partner projects will maintain their
independent organization frameworks and have ownership of the
software they are developing and deploying utilizing the common
standards.  The HSC will publish the code that complies with the
standards and organize peer-reviews for the software that is developed
under its process and guidelines.  More specifically, the scope of
the HSC work will be to
\begin{itemize}
\item provide technical expertise in architecture and design issues,
\item propose standards for component interfaces and layout, to
enable independent development and easy component integration,
\item facilitate the development of non-domain specific components
(geometry representation, data representation, …) that can be used
by many projects,
\item develop standards and guidelines for testing releasing and
documenting software, and
\item propose solutions for distributed collaborative environment
and common infrastructure such as build tools, testing and validation
suites
\end{itemize}

In order for this program of work to be successful, the Consortium
will adopt an “open-source” software model and rely on agile
development methods.  The organization framework of the Consortium
will include Technical Domain Forums each with responsibility in a
technical domain such as simulation, analytics, frameworks, and
crosscutting R\&D activities (the concurrency forum is a good example).
The Forums will facilitate the organization of workshops and meetings,
document and distribute the findings (wikis, blogs, etc).   A
possible organization chart for the Consortium is shown below

Is a picture necessary?

Role Definitions

Advisory Board: consists of representatives from the user community
(HEP experiments, non-HEP in future), partner projects, and software
industry.  The role of the Advisory Board is to provide feedback
and recommendations on scientific and technical priorities.

Steering Council: consists of representatives of participating
institutions and projects.  The role of the council is to identify
common areas of interest and discuss common goals and overall
direction of the HSC in the best interest of the Consortium.

Software Engineering Board: software experts with expertise covering
all project domain areas. The participating institutions and partner
projects contribute the members of this board.  Provide expertise
and advise on software architecture and engineering, contribute to
peer-reviews, discuss and propose best practices and guidelines.
The members of the board play a leading role in the activities of
the Technical Domain Forums.

Technical Domain Forums: provide coordination and facilitate
communication between the participating projects and the consortium.
Organize workshops; maintain wikis, blogs, etc.


Summary, if needed???

A HEP Software Consortium will provide the means to leverage and
coordinate HEP ongoing software activities.  The proposed HSC will:
\begin{itemize}
\item your favorite bullets,
\item provide the means to recognize and encourage the common aspects
of successful projects, facilitate collaboration between partner
projects, and facilitate adoption of successful software engineering
practices,
\item eventually enable contact with other scientific fields,
providing access to either resources or software tools and expertise
\end{itemize}



\newpage

%\bibliographystyle{unsrt}
%\bibliography{hsc2014}

\end{document}
